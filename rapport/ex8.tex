\paragraph{Question 1}

\begin{itemize}
\item Pour le premier produit, on doit effectuer $P_{k-1}.P_k.P_{k+1}
  + P_{k-2}.P_{k-1}.P_{k+1} + \dots + P_2.P_3.P_{k+1} +
  P_1.P_2.P_{k+1}$ opérations. Donc $O(k^4)$.
\item Pour le parenthésage symétrique, on doit effectuer $P_1.P_2.P_3
  + P_1.P_3.P_4 + P_1.P_4.P_5 + \dots + P_1.P_{k-1}.P_{k} +
  P_1.P_k+P_{k+1}$ opérations. Donc $O(k^3)$.
\end{itemize}

On remarque que le \textit{coefficient} le plus au bout de la matrice
la plus imbriquée (càd $p_{k+1}$ pour le premier produit, et $P_1$
pour le second) est \textit{reporté} à chaque \textit{séquence} de
calcul, et que l'on peut donc factoriser le calcul global par ce
terme. Par conséquent, selon la valeur du terme ainsi répété en
fonction du parenthésage choisi, le nombre d'opérations pour un même
produit de matrices peut énormément varier.

\paragraph{Question 2}

Montrons que le nombre $c(k)$ de parenthésages possibles d'un produit
de $k$ matrices vérifie $\sum_{i=1}^{k-1}c(i).c(k-i)$ en posant
$c(1)=1$. Procédons par récurrence.

\begin{proof}
\begin{itemize}
\item Si l'on multiplie deux matrices, il est clair de voir que nous
  n'avons qu'un seul parenthésage possible. Si l'on applique la
  formule, alors $c(2) = c(1).c(2-1) = c(1).c(1) = 1$. La propriété
  est donc vérifiée pour deux matrices.
\item Supposons désormais que la propriété $c(k) =
  \sum_{i=1}^{k-1}c(i).c(k-i)$ soit vérifiée pour $k$ matrices, et
  montrons qu'elle l'est alors aussi pour $k+1$ matrices. Le fait de
  rajouter une matrice implique de rajouter un certain nombre de
  possibilités de parenthésages possibles. Étudions les~: on rajoute
  autant de parenthésages que de parenthésages existant (une
  parenthèse de plus pour englober ce qui existe déjà) pour chaque
  possibilité de parenthésage. On a donc, $c(k+1)
  =\sum_{i=1}^{k-1}c(i).c(k-i)+c(k).c(k+1-i)= \sum_{i=1}^{k}c(i).c(k+1-i)$.
\end{itemize}
\end{proof}

\paragraph{Question 3}

Afin d'appliquer les principes de la programmation dynamique pour ce
problème (trouver $m_{1,k}$ le coût minimum du produit $M_{1k}$ ainsi
que le parenthésage correspondant), nous décomposons le produit en
deux sous-produits (que l'on peut calculer par récurrence) dont nous
sommons les coûts et auquel nous rajoutons le coût de la
mutliplication des deux matrices ainsi engendrées.

\begin{equation}
\begin{cases}
m_{i,j} = \min \{(m_{i,p}+m_{p+1,j}) + (i \times p+1 \times j+1) \} \\
i \leq p \leq j \\
\end{cases}
\end{equation}

On obtient le parenthésage correspondant à chaque bloc en retenant les
$p$ (une parenthèse à la fin de chaque $p$ et au début de chaque $p+1$).

\paragraph{Question 4}

On a $M_{15} = M_1.M_2.\dots M_5$. \\

On propose donc le tableau suivant pour calculer $m_{1,5}$~:
\begin{center}
\begin{tabular}{|l|c|c|c|c|c|}
\hline  $i$ $\diagdown$ $k$ & $1$  & $2$ & $3$ & $4$ & $5$ \\
\hline $1$ & $1$ & $0$ & $0$ & $0$ & $0$ \\
\hline $2$ & $0$ & $1$ & $0$ & $0$ & $0$ \\
\hline $3$ & $0$ & $0$ & $1$ & $0$ & $0$ \\
\hline $4$ & $0$ & $0$ & $0$ & $1$ & $0$ \\
\hline $5$ & $0$ & $0$ & $0$ & $0$ & $1$ \\
\hline
\end{tabular}
\end{center}

TODO

