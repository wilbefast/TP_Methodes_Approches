\paragraph{Question 1}

\begin{itemize}
\item Pour le premier produit, on doit effectuer $P_{k-1}.P_k.P_{k+1}
  + P_{k-2}.P_{k-1}.P_{k+1} + \dots + P_2.P_3.P_{k+1} + P_1.P_2.P_{k+1}$ opérations.
\item Pour le parenthésage symétrique, on doit effectuer $P_1.P_2.P_3
  + P_1.P_3.P_4 + P_1.P_4.P_5 + \dots + P_1.P_{k-1}.P_{k} + P_1.P_k+P_{k+1}$ opérations.
\end{itemize}

On remarque que le \textit{coefficient} le plus au bout de la matrice
la plus imbriquée (càd $p_{k+1}$ pour le premier produit, et $P_1$
pour le second) est \textit{reporté} à chaque \textit{séquence} de
calcul, et que l'on peut donc factoriser le calcul global par ce
terme. Par conséquent, selon la valeur du terme ainsi répété en
fonction du parenthésage choisi, le nombre d'opérations pour un même
produit de matrices peut énormément varier.

\paragraph{Question 2}

Montrons que le nombre $c(k)$ de parenthésages possibles d'un produit
de $k$ matrices vérifie $\sum_{i=1}^{k-1}c(i).c(k-i)$ en posant
$c(1)=1$. Procédons par récurrence.

\begin{proof}
\begin{itemize}
\item Si l'on multiplie deux matrices, il est clair de voir que nous
  n'avons qu'un seul parenthésage possible. Si l'on applique la
  formule, alors $c(2) = c(1).c(2-1) = c(1).c(1) = 1$. La propriété
  est donc vérifiée pour deux matrices.
\item Supposons désormais que la propriété $c(k) =
  \sum_{i=1}^{k-1}c(i).c(k-i)$ soit vérifiée pour $k$ matrices, et
  montrons qu'elle l'est alors aussi pour $k+1$ matrices. Le fait de
  rajouter une matrice implique de rajouter un certain nombre de
  possibilités de parenthésages possibles. Étudions les~: on rajoute
  autant de parenthésages que de parenthésages existant (une
  parenthèse de plus pour englober ce qui existe déjà) pour chaque
  possibilité de parenthésage. On a donc, $c(k+1)
  =\sum_{i=1}^{k-1}c(i).c(k-i)+c(k).c(k+1-i)= \sum_{i=1}^{k}c(i).c(k+1-i)$.
\end{itemize}
\end{proof}
A VERIFIER ENSEMBLE

\paragraph{Question 3}

TODO

\paragraph{Question 4}

TODO

