\subsection{Algorithme $\frac{3}{2}$ approché}

Nous étudierons ici un exemple d'algorithme approché pour le problème du 
Voyageur de Commerce qui renvoie des tour pas plus qu'une fois et demi plus
couteux que le tour optimal. Nous parlons donc de \og $\frac{3}{2}$ approximation \fg{}.

\begin{algorithm}[!ht]
\caption{Approximation $\frac{3}{2}$ pour le TSP}
\label{3-2tsp}
\begin{algorithmic}[1]
\REQUIRE un graphe $G$
\STATE faire ACPM $T$ $G$ de coût $w$
\STATE $W := $ sommets de degré impair de $T$
\STATE chercher un couplage $M$ de $G$ poids min $m$ saturant tous les
sommets de $W$
\STATE $G' := $ le graphe construit à partir des arêtes de $M$ et de
$T$
\STATE Soit P un parcours eulérien de longueur $p$, on a $p:= w + m$
\STATE retourner un cycle de longueur $c \leq p$
\end{algorithmic}
\end{algorithm}

Il s'agit de l'algorithme de Nicos Christofides, découvert en 1976.

\subsection{Implémentation}

Nous n'avons pas implémenté cet algorithme. Néanmoins, nous aurions
probablement utilisé le langage C/C++ avec la bibliothèque
boost::graph pour cela. De plus, nous aurions mesuré les temps
d'exécution de nos fonctions à l'aide de l'outil de profiling gprof.
