Nous nous intéresserons au sein de cette section à la programmation de
la méthode de Branch and Bound pour résoudre le problème du Voyageur
de Commerce (TSP).

\subsection{Détermination de la solution initiale}

Nous nous intéresserons à trois heuristiques en particulier.
\begin{itemize}
\item Chaîne de poids le plus faible et ajout d'une arête pour le
  cycle
\item Voisinage 2--opt
\item Voisinage 3--opt
\end{itemize}

\subsection{Algorithme}

\subsubsection{Pseudo-code}

Nous proposons d'implémenter l'algorithme récursif présenté ci-dessous.

\begin{algorithm}[!ht]
\caption{Branch and Bound pour le TSP}
\label{BBtsp}
\begin{algorithmic}[1]
\REQUIRE un sommet racine, une borne inférieure, un graphe $G$
\STATE faire ACPM $G - \{x \}$
\STATE relier $x$ à ses deux voisins les plus proches en terme de coût
\IF{valeur du cycle > borne inférieure}
\IF{tous les sommets sont de degré $2$}
\STATE mettre à jour la solution courante
\ELSE
\STATE choisir un sommet $y$ de degré $\geq$ 2
\FOR{chaque arête $e_i$ issue de $y$}
\STATE $G' := G - \{ e_i \}$
\STATE BranchAndBound($y$, borne inférieure, $G'$)
\ENDFOR
\ENDIF
\ENDIF
\STATE retourner la plus petite des solutions acceptables
\end{algorithmic}
\end{algorithm}

La borne inférieure est une solution initiale obtenue grâce à une heuristique.

\subsubsection{Utilisation du branch and bound pour le TSP}

La technique du branch and bound a permis dans les années 1980
d'obtenir une nouvelle étape dans la course au record du nombre de
villes pour le calcul d'une solution, ici 2392 villes (le record en
2006 était de 85900 villes, alors qu'avec la programmation dynamique
que nous avons implémentée, nous ne pouvons dépasser une vingtaine de
villes).

\subsection{Implémentation}

Nous n'avons pas pu mener de tests sur le branch and bound pour le tsp
suite à des erreurs de compilation non réglées à ce jour. Cependant,
nous avons tenté de proposer une implémentation de l'algorithme décrit plus haut
en Ocaml (à l'aide de la bibliothèque Graph Pack), dont voici un extrait~:
\begin{lstlisting}
let rec branchBound g x borneInf solution = 
	       let g' = copy g in
	       remove_vertex g x
	       let arbre = spanningtree g in
	       let cycle = relier x g' g in 
	       let val = valeurCycle cycle in
	       if val > borneInf then 
		 if tousDegre2 cycle then
		   solution = val
		 else
		   let y = choixSup2 g in
		   let aretesSucc = succ_e g y in 
		   let listeSolutions = map branche aretesSucc in
	       else()
		 solution = min listeSolutions
\end{lstlisting}

