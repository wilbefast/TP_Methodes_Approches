\paragraph{Modélisation}

Le problème de la partition se modélise de la manière suivante~:
\begin{equation}
\sum_{a \in A'} p(a)= \sum_{a \in A
    \backslash A'}p(a)
\end{equation}

\paragraph{Formules de programmation dynamique}

TODO : à reprendre \\

On se basera sur la récurrence suivante~:

\begin{equation}
\begin{cases}
tab[0] = 1; \\
si ( tab[j] == 1 ) {
       tab[j + poids[i]] = 1;
     } \\
\end{cases}
\end{equation}

\paragraph{Implémentation en C}

Nous présentons donc l'algorithme suivant, qui est une sorte de
restriction (cas particulier) de l'algorithme proposé pour la
résolution du problème du sac à dos~: \\

TODO

\paragraph{Complexité théorique}

Soit $N$ la somme des poids.
La complexité en temps de notre algorithme est $O(nN)$.
La complexité mémoire est $O(N).$

\paragraph{Jeux de tests et temps d'exécution}

Nous présentons ici des jeux de tests pour le problème partition en variant à chaque fois le nombre d'objets et en mesurant le temps d'exécution moyen sur 100 tests.
\begin{table}[h!]
\centering
\begin{tabular}{|c|c|}
\hline
Nombre d'objets & temps d'exécution moyen en ms\\
\hline
150 & 1\\
\hline
500 & 14\\
\hline
650 & 24\\
\hline
800 & 32\\
\hline
950 & 43\\
\hline
1000 & 51\\
\hline
2000 & 185\\
\hline
3000 & 565\\
\hline
5000 & 1536\\
\hline
\end{tabular}
\caption {Variation du temps d'exécution en fonction du nombre d'objets}
\end{table}\\
La variation du temps d'exécution en fonction du nombre d'objets est donnée par la courbe ci-dessous~:
\begin{figure}[h!]
\centering
\begin{tikzpicture}[scale=1.2]
    \begin{axis}[title=Jeux de tests pour partition, xlabel= nombre d'objets, ylabel= temps d'exécution]
      \addplot
        table[col sep=comma]{../charts/partition.csv};
        \legend{exécution de partition}
    \end{axis}
\end{tikzpicture}
\caption{Temps d'exécution de partition.}
\end{figure}

\paragraph{Conclusion sur la complexité}