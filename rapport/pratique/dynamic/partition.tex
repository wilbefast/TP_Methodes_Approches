%-------------------------------------------------------------------------------
% DEFINITIONS
%-------------------------------------------------------------------------------
\subsection{Définition du problème}

Le problème de Partition est un problème de décision~: nous voulions savoir si ou non il est possible étant donnée un ensemble $A$ de valeurs de diviser $A$ en deux de telle manière à ce que la somme des valeurs de part et d'autre soit la même. La formule \ref{partition_equation} est une modélisation mathématique du problème.

\begin{equation}
\sum_{a \in (A' \subset A) } = \sum_{b \in (A \backslash A')}
\label{partition_equation}
\end{equation}

%-------------------------------------------------------------------------------
% REDUCTION POLYNOMIALE
%-------------------------------------------------------------------------------
\subsection{Réduction polynomiale}

Il est possible de résoudre Partition en utilisant un algorithme conçu pour résoudre Sac à Dos. Il suffit de passer de l'instance de Partition à une instance de Sac à Dos de capacité maximum $\frac{|A|}{2}$ en associant à chaque valeur $v$ un objet d'utilité $1$ et de poids $v$.

L'algorithme du Sac à Dos rempli totalement le sac si et seulement si l'instance original peut être divisé en deux. Nous utiliserons cependant un algorithm spécialisé d'une part parce que les réductions polynomiales restent polynomiales (donc autant prendre un algorithme polynomiale de basse) et d'autre du fait que nous n'y avions pas pensés à l'époque.  

%-------------------------------------------------------------------------------
% TECHNIQUE
%-------------------------------------------------------------------------------
\subsection{Technique de résolution}

Soit $N$ la somme des valeurs de $A$. Ce que nous voulions savoir est si ou non nous pouvons séparer $A$ en deux sous-ensemble de valeur totale $\frac{N}{2}$. 

L'astuce est d'utiliser un tableau de booléens \emph{somme} dont les indices correspondent à des sommes allant de $0$ à $N$. $somme[s] = \top$ si et seulement si nous sommes capables ajouter des valeurs de $A$ ensemble de tel manière à atteindre la valeur totale $s$. 

Nous nous intéressons finalement au contenu de la case $N/2$ de notre tableau, $somme[N/2]$, mais de base nous ne connaissons que la valeur de $somme[0]$ (toujours vrai). L'équation \ref{partition_rec} nous permet d'étendre par récurrence notre raisonnement de de la somme nulle (qui est toujours une possibilité) jusqu'à la moitié de la somme totale des valeurs.

\begin{equation}
\label{partition_rec}
somme[s] = 
\begin{cases}
\top \text{ si } s = 0; \\
somme[s] \rightarrow \forall a \in \text{ A } somme[s + a] = \top \text{ sinon} ;
\end{cases}
\end{equation}

%-------------------------------------------------------------------------------
% ALGORITHME
%-------------------------------------------------------------------------------
\subsection{Algorithme}
L'algorithme \ref{dp_partition} applique l'équation \ref{partition_rec} pour résoudre Partition.


\begin{algorithm}[!ht]
\caption{DP Partition}
\label{dp_partition}
\begin{algorithmic}[1]
\REQUIRE $n$ values $v_1 \dots v_n$
\STATE $total := 0$
\FOR{i from 1 to n}
	\STATE $total := total + v_i$
\ENDFOR
\IF{$total \equiv 1 [2]$}
	\RETURN $\perp$
\ENDIF
\STATE $can\_sum[0] := \top$ 
\FOR{i from 1 to n}
	\FOR{v from total/2 to 1}
		\IF{$can\_sum[i]$}
			\STATE $can\_sum[v + v_i] := \top$
		\ENDIF
	\ENDFOR
\ENDFOR
\RETURN $can\_sum[total/2]$
\end{algorithmic}
\end{algorithm}

Nous devions allouer un tableau de taille $\frac{N}{2}$, donc la complexité mémoire de l'algorithme est de $O(N).$ Il faut parcourir ce tableau en entier pour chaqu'un des n valeurs, donc la complexité en temps de calcul est $O(nN)$. 


%-------------------------------------------------------------------------------
% IMPLEMENTATION
%-------------------------------------------------------------------------------
\subsection{Implémentation en \texttt{C}}
\label{implementation_c}
Il est important de noter l'allocation dynamique ligne 17 qui peut très facilement échouer avec des valeur choisies aléatoirement. La fonction \texttt{rand()} du C rend un entier entre $0$ et $2^{32}$, donc avec $5000$ objets nous pouvions potentiellement allouer un tableau de $10^{13}$ booléens, soit 78 téraoctets de RAM.

Malheureusement notre machine de tests n'avant que 4 gigaoctets de RAM à disposition, donc des précautions durent être prises :

\vspace{0.5cm}

\lstinputlisting[language=C,morekeywords={}]{../code/partition.c} 

%-------------------------------------------------------------------------------
% TESTS
%-------------------------------------------------------------------------------
\subsection{Tests et conclusion}

Pour les raisons explicités dans la section \ref{implementation_c} nous avions du borner les valeurs générés aléatoirement~: $50$ fut choisie comme valeur maximum. Encore une fois les valeur rapportés dans les figures \ref{partition_courbe} et \ref{partition_table} sont des moyennes pour $100$ testes.

\begin{figure}[ht]
% LEFT-HAND SIDE
\begin{minipage}[b]{0.5\linewidth}
\centering
\begin{tikzpicture}[scale=0.9]
    \begin{axis}[title=Jeux de tests pour partition, xlabel= nombre d'objets, ylabel= temps d'exécution]
      \addplot
        table[col sep=comma]{../charts/partition.csv};
        %\legend{exécution de partition}
    \end{axis}
\end{tikzpicture}
\caption {Courbe des temps d'exécution pour Partition}
\label{partition_courbe}
\end{minipage}
% RIGHT-HAND SIDE
\hspace{0.5cm}
\begin{minipage}[b]{0.5\linewidth}
\centering
\begin{tabular}[scale=0.9]{|c|c|}
\hline
Nombre d'objets & t moyenne en ms\\
\hline
150 & 1\\
\hline
500 & 14\\
\hline
650 & 24\\
\hline
800 & 32\\
\hline
950 & 43\\
\hline
1000 & 51\\
\hline
2000 & 185\\
\hline
3000 & 565\\
\hline
5000 & 1536\\
\hline
\end{tabular}
\caption {Tableau des temps d'exécution pour Partition}
\label{partition_table}
\end{minipage}
\end{figure}

Pour comprendre cette courbe quadratique il faut se rappeler que les valeurs sont générés aléatoirement  entre $0$ et $50$, donc en moyenne la somme des valeurs $N$ est de $25n$. De ce fait $O(nN)$ correspond à $O(n^2)$, donc exactement ce que nous voyons apparaitre lors des tests.
