%-------------------------------------------------------------------------------
% DEFINITIONS
%-------------------------------------------------------------------------------
\subsubsection{Définitions}

Le problème de Sac à Dos, ou \og Knapsack \fg{} en Anglais, s'annonce de la manière suivante : nous avons un ensemble d'objets aillant chaqu'un un poids (resp. volume) et une valeur d'utilité. Nous devions choisir parmi ces objets un sous-ensemble donc le poids (resp. volume) totale ne dépasse pas un certain seuil (la capacité de notre sac), mais de tel manière que l'utilité totale soit maximisé.

Ce problème se modélise en programmation linéaire de la
manière suivante, où b la capacité du sac, $c_i$ l'utilité et $a_i$ de l'objet i, et $x_i = 1$ si l'objet i est prise et $0$ sinon ~:

\begin{equation}
\begin{cases}
Max~z=\sum_{i=1}^nc_ix_i \\
\sum_{i=1}^na_ix_i \leq b \\
x_i \in\{0, 1\}, i=1\dots n\\
\end{cases}
\end{equation}

Le problème de Sac à Dos apparait souvent dans le réel : il existe un grand nombre d'applications économiques (ex: choix d'investissements) et industrielles (ex: découpage de minerai).

%-------------------------------------------------------------------------------
% FORMULAS
%-------------------------------------------------------------------------------
\subsubsection{Formule de récurrence}

Par la suite nous utiliserons les notations suivantes :

\begin{itemize}
\item \textbf{P} : la capacité maximum de notre sac,
\item \textbf{n} : le nombre d'objet,
\item \textbf{poids[i]} : le poids de la $i^{i+1\grave{e}me}$ objet,
\item \textbf{utilité[i]} : l'utilité de la $i^{i+1\grave{e}me}$ objet,
\item \textbf{optimale[i][p]} : l'utilité maximum atteignable en utilisant les $i$ premières objets et en respectant un poids totale de $p$.
\end{itemize} 

Nous cherchions à connaitre optimale[n][P]. Nous verrons par la suite qu'il est possible de calculer cette valeur à partir de optimale[n-1][P] et optimale[n-1][P-poids[i]], e cetera de manière récursive.

Nous construirons donc un tableau de résultats intermédiaires ligne par ligne et colonne par colonne. La première ligne du tableau est rempli à la main : évidement l'utilité maximum atteignable qu'avec l'objet $0$ est son utilité à lui, sous réserve que le poids que nous nous permettons est en dessus de $poids[0]$ :

\begin{equation}
optimale[0][p] =
	\begin{cases}
		0 \text{ si } p < poids[0];	\\
		utilit\acute{e}[i] \text{ sinon};
	\end{cases} \\
\end{equation}

Si nous ne prenons pas l'objet $i$ alors nous n'augmentons pas l'utilité maximum par rapport au maximum atteignable avec l'ensemble des objets $0 \dots i-1$. Nous nous définissons donc l'utilité maximale en laissant l'objet $i$ :

\begin{equation}
laisser(i, p) = optimale[i-1][p];
\end{equation}

Si par contre l'objet $i$ est prise il faut s'assurer qu'il y ait de la place pour lui, donc nous ajoutons son utilité à l'utilité maximum atteignable avec les objets $0 \dots i-1$ et un poids inférieure ou égale à $p-poids[i]$. Ainsi l'utilité maximale si nous prenons l'objet $i$ est la suivante :

\begin{equation}
prendre(i, p) = optimale.t[i-1][p - poids[i]] + utilit\acute{e}[obj];
\end{equation}

Nous pouvions finalement remplir notre tableau de manière récursive, pour $i \neq 0$ :

\begin{equation}
optimale[i][p] =
	\begin{cases}
		laisser(i, p) \text{ si } p < poids[i];	\\
		max(prendre(i, p), laisser(i, p)) \text{ sinon};
	\end{cases}
\end{equation}

%-------------------------------------------------------------------------------
% IMPLEMENTATION
%-------------------------------------------------------------------------------
\subsubsection{Implémentation en C}

\lstinputlisting[language=C,morekeywords={}]{../code/knapsack.c} 


%-------------------------------------------------------------------------------
% COMPLEXITÉ
%-------------------------------------------------------------------------------
\subsubsection{Complexité}

La complexité en temps de notre algorithme est en $O(nW)$. Il en va de
même pour la complexité mémoire.

%-------------------------------------------------------------------------------
% TESTS
%-------------------------------------------------------------------------------
\subsubsection{Jeux de tests}

Nous présentons ici des jeux de tests pour le problème sac à dos en variant à chaque fois le nombre d'objets et en mesurant le temps d'exécution moyen sur 100 tests et pour une capacité de 5000 .
\begin{table}[h!]
\centering
\begin{tabular}{|c|c|}
\hline
Nombre d'objets & temps d'exécution moyen en ms\\
\hline
100 & 8\\
\hline
150 & 12\\
\hline
200 & 16\\
\hline
300 & 24\\
\hline
500 & 40\\
\hline
800 & 65\\
\hline
1000 & 82\\
\hline
1200 & 101\\
\hline
2000 & 165\\
\hline
3000 & 246\\
\hline
 4000 & 329\\
\hline
  5000 & 410\\
\hline
\end{tabular}
\caption {Variation du temps d'exécution en fonction du nombre d'objets}
\end{table}\\
La variation du temps d'exécution en fonction du nombre d'objets est donnée par la courbe ci-dessous~:
\begin{figure}[h!]
\centering
\begin{tikzpicture}[scale=1.2]
    \begin{axis}[title=Jeux de tests pour Sac à dos, xlabel= nombre d'objets, ylabel= temps d'exécution]
      \addplot
        table[col sep=comma]{../charts/sac.csv};
        \legend{exécution de sac à dos}
    \end{axis}
\end{tikzpicture}
\caption{Temps d'exécution de sac à dos.}
\end{figure}

Nous présentons ici des jeux de tests pour le problème sac à dos en variant à chaque fois la capacité du sac et en mesurant le temps d'exécution moyen sur 100 tests et pour 100 objets.
\begin{table}[h!]
\centering
\begin{tabular}{|c|c|}
\hline
Capacité & temps d'exécution moyen en ms\\
\hline
500 & 0\\
\hline
1000 & 1\\
\hline
2000 & 3\\
\hline
2500 & 3\\
\hline
3000 & 4\\
\hline
4000 & 6\\
\hline
5000 & 8\\
\hline
10000 & 16\\
\hline
\end{tabular}
\caption {Variation du temps d'exécution en fonction du capacité du sac}
\end{table}\\
La variation du temps d'exécution en fonction du capacité du sac est donnée par la courbe ci-dessous~:
\begin{figure}[h!]
\centering
\begin{tikzpicture}[scale=1.2]
    \begin{axis}[title=Jeux de tests pour Sac à dos, xlabel= capacité, ylabel= temps d'exécution]
      \addplot
        table[col sep=comma]{../charts/sac1.csv};
        \legend{exécution de sac à dos}
    \end{axis}
\end{tikzpicture}
\caption{Temps d'exécution de sac à dos.}
\end{figure}

\subsubsection{Conclusion sur la complexité}
