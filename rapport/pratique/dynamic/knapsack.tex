Nous avons commencé par nous intéresser au problème du sac à dos avant
partition car celui-ci est le plus général.

%-------------------------------------------------------------------------------
% MODELISATION
%-------------------------------------------------------------------------------
\subsubsection{Modélisation}

Le problème du sac à dos se modélise en programmation linéaire de la
manière suivante~:

\begin{equation}
\begin{cases}
Max~z=\sum_{i=1}^nc_ix_i \\
\sum_{i=1}^na_ix_i \leq b \\
x_i \in\{0, 1\}, i=1\dots n\\
\end{cases}
\end{equation}

%-------------------------------------------------------------------------------
% FORMULAS
%-------------------------------------------------------------------------------
\subsubsection{Formules de programmation dynamique}

TODO : à reprendre

Les formules de programmation dynamique pour le problème du sac à dos
sont les suivantes~:
\begin{equation}
\begin{cases}
tab[0][w] = 0 \\
tab[i][j] = max(tab[i-1] [j], (tab[i-1] [j-poids[i]] + utilite[i])); \\
\end{cases}
\end{equation}

%-------------------------------------------------------------------------------
% IMPLEMENTATION
%-------------------------------------------------------------------------------
\subsubsection{Implémentation en C}

Nous avons donc implémenté l'algorithme suivant. \\


%-------------------------------------------------------------------------------
% COMPLEXITÉ
%-------------------------------------------------------------------------------
\subsubsection{Complexité}

La complexité en temps de notre algorithme est en $O(nW)$. Il en va de
même pour la complexité mémoire.

%-------------------------------------------------------------------------------
% TESTS
%-------------------------------------------------------------------------------
\subsubsection{Jeux de tests}

Nous présentons ici des jeux de tests pour le problème sac à dos en variant à chaque fois le nombre d'objets et en mesurant le temps d'exécution moyen sur 100 tests et pour une capacité de 5000 .
\begin{table}[h!]
\centering
\begin{tabular}{|c|c|}
\hline
Nombre d'objets & temps d'exécution moyen en ms\\
\hline
100 & 8\\
\hline
150 & 12\\
\hline
200 & 16\\
\hline
300 & 24\\
\hline
500 & 40\\
\hline
800 & 65\\
\hline
1000 & 82\\
\hline
1200 & 101\\
\hline
2000 & 165\\
\hline
3000 & 246\\
\hline
 4000 & 329\\
\hline
  5000 & 410\\
\hline
\end{tabular}
\caption {Variation du temps d'exécution en fonction du nombre d'objets}
\end{table}\\
La variation du temps d'exécution en fonction du nombre d'objets est donnée par la courbe ci-dessous~:
\begin{figure}[h!]
\centering
\begin{tikzpicture}[scale=1.2]
    \begin{axis}[title=Jeux de tests pour Sac à dos, xlabel= nombre d'objets, ylabel= temps d'exécution]
      \addplot
        table[col sep=comma]{../charts/sac.csv};
        \legend{exécution de sac à dos}
    \end{axis}
\end{tikzpicture}
\caption{Temps d'exécution de sac à dos.}
\end{figure}

Nous présentons ici des jeux de tests pour le problème sac à dos en variant à chaque fois la capacité du sac et en mesurant le temps d'exécution moyen sur 100 tests et pour 100 objets.
\begin{table}[h!]
\centering
\begin{tabular}{|c|c|}
\hline
Capacité & temps d'exécution moyen en ms\\
\hline
500 & 0\\
\hline
1000 & 1\\
\hline
2000 & 3\\
\hline
2500 & 3\\
\hline
3000 & 4\\
\hline
4000 & 6\\
\hline
5000 & 8\\
\hline
10000 & 16\\
\hline
\end{tabular}
\caption {Variation du temps d'exécution en fonction du capacité du sac}
\end{table}\\
La variation du temps d'exécution en fonction du capacité du sac est donnée par la courbe ci-dessous~:
\begin{figure}[h!]
\centering
\begin{tikzpicture}[scale=1.2]
    \begin{axis}[title=Jeux de tests pour Sac à dos, xlabel= capacité, ylabel= temps d'exécution]
      \addplot
        table[col sep=comma]{../charts/sac1.csv};
        \legend{exécution de sac à dos}
    \end{axis}
\end{tikzpicture}
\caption{Temps d'exécution de sac à dos.}
\end{figure}

\subsubsection{Conclusion sur la complexité}
