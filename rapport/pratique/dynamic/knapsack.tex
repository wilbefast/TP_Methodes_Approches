%-------------------------------------------------------------------------------
% DEFINITIONS
%-------------------------------------------------------------------------------
\subsection{Définition du problème}

Le problème de Sac à Dos, ou \og Knapsack \fg{} en Anglais, s'annonce de la manière suivante : nous avons un ensemble d'objets aillant chaqu'un un poids (resp. volume) et une valeur d'utilité. Nous devions choisir parmi ces objets un sous-ensemble donc le poids (resp. volume) totale ne dépasse pas un certain seuil (la capacité de notre sac), mais de tel manière que l'utilité totale soit maximisé.

Ce problème se modélise en programmation linéaire de la
manière suivante, où b la capacité du sac, $c_i$ l'utilité et $a_i$ de l'objet i, et $x_i = 1$ si l'objet i est prise et $0$ sinon ~:

\begin{equation}
\begin{cases}
Max~z=\sum_{i=1}^nc_ix_i \\
\sum_{i=1}^na_ix_i \leq b \\
x_i \in\{0, 1\}, i=1\dots n\\
\end{cases}
\end{equation}

Le problème de Sac à Dos apparait souvent dans le réel : il existe un grand nombre d'applications économiques (ex: choix d'investissements) et industrielles (ex: découpage de minerai).

%-------------------------------------------------------------------------------
% NOTATIONS
%-------------------------------------------------------------------------------
\subsection{Notations}

Par la suite nous utiliserons les notations suivantes :

\begin{itemize}
\item \textbf{P} : la capacité maximum de notre sac,
\item \textbf{n} : le nombre d'objet,
\item \textbf{poids[i]} : le poids de la $i^{i+1\grave{e}me}$ objet,
\item \textbf{utilité[i]} : l'utilité de la $i^{i+1\grave{e}me}$ objet,
\item \textbf{optimale[i][p]} : l'utilité maximum atteignable en utilisant les $i$ premières objets et en respectant un poids totale de $p$.
\end{itemize} 

Nous cherchions à connaitre optimale[n][P]. Nous verrons par la suite qu'il est possible de calculer cette valeur à partir de optimale[n-1][P] et optimale[n-1][P-poids[i]], e cetera de manière récursive.

%-------------------------------------------------------------------------------
% FORMULAS
%-------------------------------------------------------------------------------
\subsection{Formule de récurrence}
Nous construirons donc un tableau de résultats intermédiaires ligne par ligne et colonne par colonne. La première ligne du tableau est rempli à la main : évidement l'utilité maximum atteignable qu'avec l'objet $0$ est son utilité à lui, sous réserve que le poids que nous nous permettons est en dessus de $poids[0]$ :

\begin{equation}
optimale[0][p] =
	\begin{cases}
		0 \text{ si } p < poids[0];	\\
		utilit\acute{e}[i] \text{ sinon};
	\end{cases} \\
\end{equation}

Si nous ne prenons pas l'objet $i$ alors nous n'augmentons pas l'utilité maximum par rapport au maximum atteignable avec l'ensemble des objets $0 \dots i-1$. Nous nous définissons donc l'utilité maximale en laissant l'objet $i$ :

\begin{equation}
laisser(i, p) = optimale[i-1][p];
\end{equation}

Si par contre l'objet $i$ est prise il faut s'assurer qu'il y ait de la place pour lui, donc nous ajoutons son utilité à l'utilité maximum atteignable avec les objets $0 \dots i-1$ et un poids inférieure ou égale à $p-poids[i]$. Ainsi l'utilité maximale si nous prenons l'objet $i$ est la suivante :

\begin{equation}
prendre(i, p) = optimale.t[i-1][p - poids[i]] + utilit\acute{e}[obj];
\end{equation}

Nous pouvions finalement remplir notre tableau de manière récursive, pour $i \neq 0$ :

\begin{equation}
optimale[i][p] =
	\begin{cases}
		laisser(i, p) \text{ si } p < poids[i];	\\
		max(prendre(i, p), laisser(i, p)) \text{ sinon};
	\end{cases}
\end{equation}

%-------------------------------------------------------------------------------
% ALGORITHME
%-------------------------------------------------------------------------------
\subsection{Algorithme}

Nous pouvions finalement proposer l'algorithme suivant~:

\begin{algorithm}[!ht]
\caption{DP Knapsack}
\label{dp_knapsack}
\begin{algorithmic}[1]
\REQUIRE maximum capacity $W$, number of objects $n$, $weights[n]$, $utilities[n]$  
\FOR{$w$ from 1 to $W$}
	\IF{weights[0] $\leq$ w }
		\STATE $optimal[0][w] := utilities[0]$
	\ENDIF
\ENDFOR
\FOR{$w$ from 1 to $n$}
	\FOR{$w$ from 1 to $W$}
		\STATE leave := optimal[i-1][w]
		\STATE take := optimal[i-1][w-weights[i]] + utilities[i]
		\IF{weights[i] $\leq$ w }
			\STATE $optimal[i][w] := max(leave, take)$
		\ELSE
			\STATE $optimal[i][w] := leave$
		\ENDIF
	\ENDFOR
\ENDFOR
\RETURN optimal[n][W]
\end{algorithmic}
\end{algorithm}

Étant donnée que l'algorithme \ref{dp_knapsack} itère sur les poids et les objets, la complexité temporelle de est de $O(nW)$. Il en va de même pour sa complexité mnémonique.

%-------------------------------------------------------------------------------
% IMPLEMENTATION
%-------------------------------------------------------------------------------
\subsubsection{Implémentation en C}
Une fois énoncé l'algorithme est simple à implémenter. Nous pourtant par contre une structure matrice \texttt{matrix\_t} fait maison : 

\lstinputlisting[language=C,morekeywords={}]{../code/knapsack.c} 


%-------------------------------------------------------------------------------
% TESTS
%-------------------------------------------------------------------------------
\subsection{Tests et conclusion}

Pour vérifier que notre implémentation respecte bien la complexité théorique de $O(nW)$ nous avions lancés une batterie des tests en faisant varier : 

\begin{itemize}
\item le nombre d'objets, avec la capacité fixé à $5000$,
\item la capacité, avec un nombre d'objets fixé à $100$.
\end{itemize}

Les valeurs rapportés dans les figures \ref{knapsack_test_objects} et \ref{knapsack_test_capacity} sont des temps moyennes d'exécution pour $100$ tests. Le poids et l'utilité de chaque objet sont générés aléatoirement mais, mais les poids sont bornés par la capacité (aucun élément n'est jamais plus grand que le sac entier) :

\begin{figure}[ht]
% LEFT-HAND SIDE
\begin{minipage}[b]{0.5\linewidth}
\centering
\begin{tikzpicture}[scale=0.9]
    \begin{axis}[title=Variation du nombre d'objets, xlabel= nombre d'objets, ylabel= temps d'exécution]
      \addplot
        table[col sep=comma]{../charts/sac.csv};
        %\legend{exécution de sac à dos}
    \end{axis}
\end{tikzpicture}
\caption{Temps d'exécution de Sac à Dos avec nombre d'objets variable.}
\label{knapsack_test_objects}
\end{minipage}
% RIGHT-HAND SIDE
\hspace{0.5cm}
\begin{minipage}[b]{0.5\linewidth}
\centering
\begin{tikzpicture}[scale=0.9]
    \begin{axis}[title=Variation de la capacité, xlabel= capacité, ylabel= temps d'exécution]
      \addplot
        table[col sep=comma]{../charts/sac1.csv};
        %\legend{exécution de Sac à Dos}
    \end{axis}
\end{tikzpicture}
\caption{Temps d'exécution de Sac à Dos avec capacité variable.}
\label{knapsack_test_capacity}
\end{minipage}
\end{figure}

Notre implémentation est donc linéaire en le nombre d'objets et la capacité maximum, ce qui correspond bien à la complexité théorique.
