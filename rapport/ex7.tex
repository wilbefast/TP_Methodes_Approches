\paragraph{Sur le problème de la partition}

\begin{itemize}
\item a) La condition nécessaire sur la somme des poids des $n$ objets
  est la suivante~: on veut $\sum_{a \in A'} p(a)= \sum_{a \in A
    \backslash A'}p(a)$
\item b) 
\begin{itemize}
\item i) La formule qui lie les lignes $i$, $i-1$ et $p(a_i)$ est $A_i
  := A_{i-1} \cup A_{i, p(a_i)} \cup A_{i-1}+p(a_i)$ avec
  $A_{i-1}+p(a_i) \leq P$.
\item ii) Construisons le tableau à partir de l'exemple donné:
\begin{table}[h!]
\centering
	\begin{tabular}{|c|c|c|c|c|c|c|c|c|c|c|c|c|c|c|c|c|c|}
    \hline
     i $\diagdown$ p & 0 & 1 & 2 & 3 & 4 & 5 & 6 & 7 & 8 & 9 & 10 & 11 & 12 & 13 & 14 & 15 & 16 \\
    \hline
   1 & 0 & 0 & 0 & 0 & 0 & 1 & 0 & 0 & 0 & 0 & 0 & 0 & 0 & 0 & 0 & 0 & 0 \\
     \hline
	2& 0 & 0 & 0 & 0 & 0 & 1 & 0 & 0 & 0 & 1 & 0 & 0 & 0 & 0 & 1 & 0 & 0 \\
   \hline
	3 & 0 & 0 & 0 & 1 & 0 & 1 & 0 & 0 & 1 & 1 & 0 & 0 & 1 & 0 & 1 & 0 & 0 \\
	\hline
	4 & 0 & 0 & 0 & 1 & 0 & 1 & 0 & 0 & 1 & 1 & 0 & 1 & 1 & 1 & 1 & 0 & 0 \\
	  \hline
	5 & 0 & 0 & 1 & 1 & 0 & 1 & 0 & 1 & 1 & 1 & 1 & 1 & 1 & 1 & 1 & 0 & 0 \\
	\hline
	6 & 0 & 0 & 1 & 1 & 0 & 1 & 0 & 1 & 1 & 1 & 1 & 1 & 1 & 1 & 1 & 0 & 0 \\
	\hline
\end{tabular}
\caption {Remplissage du tableau sur l'exemple donné}
\end{table} 
\item iii) Nous proposons les deux algorithmes ci-dessous.
\begin{algorithm}[t]
\caption{Algorithme général}
\label{algoexo7}
\begin{algorithmic}[1]
\STATE $P_1 := \emptyset $,  $P_1 := \emptyset $
\FOR{$i=1 \to n$}
\WHILE{$P_1 < P$ et $P_2 < P$}
\STATE remplir $A_i$ en utilisant la formule présentée avant
\STATE faire Test
\IF {$A(i,j) == 1$}
\STATE $A(i,j):=0$
\ENDIF
\ENDWHILE
\ENDFOR
\end{algorithmic}
\end{algorithm}

\begin{algorithm}[t]
\caption{Test}
\label{algoexo7test}
\begin{algorithmic}[1]
\FOR{$j=0 \to P$}
\IF{$A(i,j) == 1$ et $P_{1}+j \leq P$}
\STATE $P_1 := P_1 \bigcup j$
\ELSE
\STATE $P_2 := P_2 \bigcup j$
\ENDIF
\ENDFOR
\end{algorithmic}
\end{algorithm}
\end{itemize}

\item c) La complexité est donc $O(n\times P)$.
\item d) Nous proposons les traces suivantes pour nos algorithmes~:
\begin{itemize}
\item La première trace que nous proposons est réalisée à partir des
  données fournies dans l'énoncé. Ici, $P=16$. On obtient ainsi $P_1 := \{ 0, 5, 9,
  2, 0\}$, $P_2 := \{ 0, 0, 3, 8, 5\}$.
\item Prenons désormais $P(a_1)=2$, $P(a_2)=4$, $P(a_3)=3$,
  $P(a_4)=1$. Ici, $P=5$. On obtient ainsi  $P_1 := \{ 0, 2, 3\}$,
  $P_2 := \{ 0, 4, 1\}$.
\item Prenons désormais $P(a_1)=2$, $P(a_2)=4$, $P(a_3)=3$,
  $P(a_4)=6$. Il n'est dans cet exemple pas possible de partager en
  deux sous-ensembles de même poids nos objets. Soit P n'existe pas et
  notre algorithme ne pourra être lancé. Soit P existe et l'algorithme
  donnera le résultat le plus proche possible pour $P_1$, et remplira
  $P_2$ avec tous les objets <<~en trop~>>.
\end{itemize}
\end{itemize}

\paragraph{Le problème du sac à dos}
\begin{itemize}
\item a) Nous justifions les formules proposées en énonçant que l'on
  obtient la solution optimale pour le problème $P(k,v)$ en calculant
  l'utilité maximale de $1$ à $k-1$ et en lui ajoutant l'utilité de
  l'objet $k$. $x_k = \frac{v}{v_k}$ représente le nombre de fois
  qu'on peut faire rentrer notre objet $k$ dans le sac. C'est pour
  cela qu'on retire $v - v_kx_k$.
\item b) Nous proposons l'exemple suivant~:
\begin{itemize}
\item Le nombres d'objets est $n = 4$
\item La capacité du sac est $V = 14$
\item $p(a_1)=2$,$p(a_2)=3$, $p(a_3)=4$, $p(a_4)=2$
\item $u(a_1)=4$,$u(a_2)=3$, $u(a_3)=2$, $u(a_4)=1$
\end{itemize}
On note ici $v_i=a_i$ le poids de l'objet $i$.

On obtient donc~:
\begin{equation}
\begin{cases}
\max z = \sum_{j=1}^3 x_ju_j \\
\sum_{j=1}^3v_jx_j \leq v - v_4x_4 \\
\end{cases}
\end{equation}

On propose donc le tableau suivant~:
\begin{center}
\begin{tabular}{|l|c|c|c|}
\hline  i $\diagdown$ p  & $p(a_i)$ & $u_i$ & $x_i$  \\
\hline $1$ & $2$ & $4$ & $3$ \\
\hline $2$ & $3$ & $3$ & $2$  \\
\hline $3$ & $4$ & $2$ & $1$ \\
\hline $4$ & $2$ & $1$ & $2$ \\
\hline
\end{tabular}
\end{center}
\vspace{1cm}
Ici, on peut prendre les objets $1$ et $3$ pour une utilité de
$3 \times 4 + 2 \times 1 = 14$.


On pourrait aussi prendre les objets $2$ et $3$ mais pour une utilité
inférieure ($8$).

Remarquons que les objets $1$ et $2$ violent la contrainte de poids.


On garde donc les objets $1$ et $3$.

En rajoutant l'utilité de l'objet $4$, on obtient $z=16$.

\item c) La complexité de l'algorithme est donc de $O(k \times V)$.
\end{itemize}

\paragraph{Le problème du voyageur de commerce}

\begin{itemize}
\item a) En appliquant la méthode présentée sur la figure 4, nous
obtenons la matrice $C$ suivante, avec respectivement pour chaque case la valeur du
chemin en cours et le dernier sommet choisi ayant mené à cette valeur~:
\begin{center}
\begin{tabular}{|l|c|c|c|c|}
\hline  card(S) $\diagdown$ i  & $1$ & $2$ & $3$ & $4$ \\
\hline $1$ & $1,\textcolor{blue}{1}$ & $2,2$ & $1,\textcolor{blue}{3}$ & $0,4$ \\
\hline $2$ & $1,\textcolor{blue}{4}$ & $3,4$ & $3,\textcolor{blue}{2}$ & $0,1$  \\
\hline $3$ & $2,\textcolor{blue}{2}$ & $3,1$ & $4,\textcolor{blue}{4}$ & $3,2$ \\
\hline $4$ & $4,\textcolor{blue}{3}$ & $8,3$ & $4,\textcolor{blue}{1}$ & $5,3$ \\
\hline $5$ & $\textcolor{red}{5},\textcolor{blue}{0}$ & $9,0$ & $\textcolor{red}{5},\textcolor{blue}{0}$ & $6,0$ \\
\hline
\end{tabular}
\end{center}
\vspace{1cm}
On obtient ainsi deux plus courts cycles hamiltoniens de longueur 5~:
\begin{itemize}
\item 0-1-4-2-3-0~;
\item 0-3-2-4-1-0.
\end{itemize}
\item b) TODO après la prog!
\end{itemize}
