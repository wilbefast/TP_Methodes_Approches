\paragraph{Sur le problème de la partition}

\begin{itemize}
\item a) La condition nécessaire sur la somme des poids des $n$ objets
  est la suivante~: on veut $\sum_{a \in A'} p(a)= \sum_{a \in A
    \backslash A'}p(a)$
\item b) 
\begin{itemize}
\item i) La formule qui lie les lignes $i$, $i-1$ et $p(a_i)$ est $A_i
  := A_{i-1} \bigcup A_{i, p(a_i)} \bigcup A_{i-1}+p(a_i)$ avec
  $A_{i-1}+p(a_i) \leq P$.
\item ii) TODO
\item iii) Nous proposons les deux algorithmes ci-dessous.
\begin{algorithm}[t]
\caption{Algorithme général}
\label{algoexo7}
\begin{algorithmic}[1]
\STATE $P_1 := \emptyset $,  $P_1 := \emptyset $
\FOR{$i=1 \to n$}
\WHILE{$P_1 < P$ et $P_2 < P$}
\STATE remplir $A_i$ en utilisant la formule présentée avant
\STATE faire Test
\IF {$A(i,j) == 1$}
\STATE $A(i,j):=0$
\ENDIF
\ENDWHILE
\ENDFOR
\end{algorithmic}
\end{algorithm}

\begin{algorithm}[t]
\caption{Test}
\label{algoexo7test}
\begin{algorithmic}[1]
\FOR{$j=0 \to P$}
\IF{$A(i,j) == 1$ et $P_{1}+j \leq P$}
\STATE $P_1 := P_1 \bigcup j$
\ELSE
\STATE $P_2 := P_2 \bigcup j$
\ENDIF
\ENDFOR
\end{algorithmic}
\end{algorithm}
\end{itemize}

\item c) La complexité est donc $O(n\times P)$.
\item d) Nous proposons les traces suivantes pour nos algorithmes~:
\begin{itemize}
\item La première trace que nous proposons est réalisée à partir des
  données fournies dans l'énoncé. Ici, $P=16$. On obtient ainsi $P_1 := \{ 0, 5, 9,
  2, 0\}$, $P_2 := \{ 0, 0, 3, 8, 5\}$.
\item Prenons désormais $P(a_1)=2$, $P(a_2)=4$, $P(a_3)=3$,
  $P(a_4)=1$. Ici, $P=5$. On obtient ainsi  $P_1 := \{ 0, 2, 3\}$,
  $P_2 := \{ 0, 4, 1\}$.
\item Prenons désormais $P(a_1)=2$, $P(a_2)=4$, $P(a_3)=3$,
  $P(a_4)=6$. Il n'est dans cet exemple pas possible de partager en
  deux sous-ensembles de même poids nos objets. Soit P n'existe pas et
  notre algorithme ne pourra être lancé. Soit P existe et l'algorithme
  donnera le résultat le plus proche possible pour $P_1$, et remplira
  $P_2$ avec tous les objets <<~en trop~>>.
\end{itemize}
\end{itemize}

\paragraph{Le problème du sac à dos}
\begin{itemize}
\item a) Nous justifions les formules proposées en énonçant que l'on
  prend le maximum des $x_ju_j$ pour $j$ de 1 à $k-1$ en enlevant le
  poids de l'objet $x_k$, et l'on rajoute à $x_ju_j$ l'utilité de
  l'objet en cours.
\item b) TODO
\item c) TODO
\end{itemize}

