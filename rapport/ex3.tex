\paragraph{Question 1}Le programme linéaire en nombres entiers qui modélise le
  problème du couplage maximum de poids minimum est le suivant.
  (notons $u_{ij}$ les arêtes du graphe)
Notons que $u_{ij} \in \{0, 1\}$ car on prend une arête ou on ne la
prend pas.

\begin{equation}
\begin{cases}
min \sum w_{ij} u_{ij} \\
\forall x_{ij} \in E ~ \sum u_{i' j'} = 1 \\
 x_{i' j'} \in adj(x_{ij}) \cup \{ x_{ij} \} \\
\end{cases}
\end{equation}

En effet, soit $\{ i, j \}$ une arête qui appartient à un couplage
$M$. Par définition, $\Gamma^{-1}(i) $ et $\Gamma^{+1}(j)$
n'appartiennent pas à $M$.

\paragraph{Question 2}

Pour obtenir un couplage de poids minimum, on doit plutôt choisir des
arêtes de valuation $\epsilon$.

\paragraph{Question 3}
\paragraph{Question 4}
\paragraph{Question 5}
La formulation initialement proposée n'est pas pertinente car elle ne
couvre pas tous les cas. On peut ainsi considérer les contre-exemples
suivants~: TODO !