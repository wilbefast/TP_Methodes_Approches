Le rapport que nous vous présentons ici reprend notre travail effectué
dans le cadre du projet du cours de méthodes approchées. Celui-ci se
compose d'une partie théorique nous laquelle nous avons résolu divers
exercices traitant de programmation linéaire, d'approximations de
problèmes NP-difficiles, mais aussi de méthodes exactes comme la
programmation dynamique ou les méthodes de branchement (Branch and
Bound et Branch and Cut), ainsi que d'une partie pratique. Dans cette
dernière, nous programmons les formules de programmation dynamique
pour trois problèmes classiques (Partition, Sac à Dos, Voyageur de
Commerce), puis nous nous intéressons plus en détail au cas du
Voyageur de Commerce avec la programmation de l'algorithme de Branch
en Bound (après application et comparaison de diverses heuristiques
pour trouver une solution initiale).

On notera en général $PL$ un programme linéaire, et $PLNE$ un
programme linéaire en nombres entiers. 

Nous désignerons également par $TSP$ le problème du voyageur de
commerce.

Rappelons enfin les quelques définitions clés suivantes~:
\begin{mydef}
Les méthodes de branch and bound sont les méthodes dites de séparation
et évaluation.
\end{mydef}

\begin{mydef}
Les méthodes de branch and cut sont les méthodes dites des coupes de
Gomory ou d'algorithmes de coupe.
\end{mydef}

\begin{mydef}
Un problème NP-Complet est un problème de décision pour la résolution duquel on ne
connait pas d'algorithme polynomial. Tous les problèmes de la classe
NP peuvent se rammener à un problème NP-complet via une réduction polynomiale.
\end{mydef}

\begin{mydef}
Un problème NP-Difficile est un problème d'optimisation qui est plus
difficile qu'un problème NP-Complet.
\end{mydef}