\paragraph{Question 1}
\begin{itemize}
\item a) $nlogn+n = O(nlogn)$
\item b) Si $j = 1$ alors $cost(T) + w_{2} \geqslant B$ et donc $cost(T) \geqslant \frac{B}{2}$.\\
Si $j \geqslant 2$ alors $\sum_{i=1}^{j} W_{j} + W_{j+1} \geqslant B$, c'est à dire $\sum_{i=1}^{j} \geqslant W_{j+1}$ et donc $cost(T) > \frac{B}{2}$.
\item c)  $s^{*} \geqslant B$ et $S \geqslant \frac{B}{2}$\\
$\frac{1}{S} \leqslant \frac{2}{B}$ et donc $\frac{S^{*}}{S} \leqslant 2$.
\end{itemize}

\paragraph{Question 2}
\begin{itemize}
\item a) L'algorithme $2$ admet une complexité de $O(n^{k+1})$ car ~:
\begin{itemize}
\item pour étendre $S$ à $S^*$ on a un coût de $O(nlogn)$~;
\item les ensembles $S$ de taille maximale $k$ sont créés par ordre lexicographique, on a
doit donc vérifier le tri à chaque création $O(nlogn)$.
\end{itemize}
\item b)
\begin{itemize}
\item i) Si $p < k$ alors on n'a pas pris tous les objets dans le
sac. Si $p = k$ c'est que tous les objets rentrent dans le sac. 
\item ii)
\begin{itemize}
\item A)
\item B)
\item C)
\item D)
\end{itemize}
\end{itemize}
\end{itemize}
