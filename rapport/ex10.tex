\paragraph{Question 1}

Le problème de bin packing est le problème de trouver un rangement
valide pour tous nos articles, qui minimise le nombre de boîtes
utilisées (avec, bien sûr, la contrainte qu'un objet n'appartienne
qu'à une seule boîte, d'où la troisième ligne de la modélisation
proposée).  On l'exprime de la manière suivante~:
\begin{equation}
\begin{cases}
min \sum_{j=1}^{n}y_j \\
\sum_{i=2}^{n} c_ix_{ij} \leq C_{y_j}, j = 1, \dots, n \\
\sum_{j=1}^{n}x_{ij}=1, i=1, \dots, n \\
x_{i,j} \in \{ 0,1 \} \\
y_j \in \{ 0,1 \} \\
\end{cases}
\end{equation}
Avec $y_j = 1$ si la boîte $j$ est utilisée, 0 sinon. \\
Avec $x_{ij} = 1 $ si article $i$ est rangé dans la boîte $j$, 0
sinon. \\
Avec $c_i$ taille de l'article $i$ et C la taille d'une boîte.


\paragraph{Question 2}
\begin{itemize}
\item a) $B = 32$, $P(a_1') = \frac{5 \times 2}{32} = \frac{10}{32}$,
  $P(a_2') = \frac{9}{16}$, $P(a_3')=\frac{6}{32}$,
  $P(a_4')=\frac{1}{2}$, $P(a_5')=\frac{4}{32}$, $P(a_6')=\frac{10}{32}$.
\item b) $B = 180$, $P(a_1') = \frac{154}{180}$,
  $P(a_2') = \frac{82}{180}$, $P(a_3')=\frac{6}{180}$,
  $P(a_4')=\frac{60}{180}$, $P(a_5')=\frac{34}{180}$, $P(a_6')=\frac{24}{180}$.
\end{itemize}

\paragraph{Question 3}
\begin{itemize}
\item Quand nous avons une instance positive, alors nous avons pour le problème Bin Packing une partition de l'ensemble d'objets en deux sous ensembles dont le poids égale à $\frac{B^{*}}{2}$. Donc, la valeur optimale pour le problème Bin Packing $m^{*} = 2$. 
\item Quand nous avons une instance négative, alors nous avons forcément un objet dont le poids dépasse $\frac{B}{2}$ et on ne nous pouvons pas le mettre ni dans la première boîte ni dans la deuxième. Donc, on doit rajouter une nouvelle boîte, d'où $m^{*} = 3$.
\end{itemize}
\paragraph{Question 4}
Nous avons un algorithme polynomial pour le problème de Bin Packing. Donc, on peut décider en temps polynomial pour le problème partition, or ce problème est NP-complet. Donc, le problème Bin Packing est non ($\frac{3}{2} - \epsilon $)--approché.


