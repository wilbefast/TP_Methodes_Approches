\documentclass[french]{beamer}
%\usepackage{listingsutf8}
%\usepackage[T1]{fontenc}
%\usepackage[utf8]{inputenc}
\usepackage{fontspec}
\usepackage[french]{babel}
\usepackage{amsmath, amssymb, mathrsfs}
\usepackage{graphicx}
\usepackage{listings}
\usepackage{tikz}
\usepackage{pgfplots}
\usepackage{epic, eepic}
\usepackage{fancybox}
\usepackage{array, multirow, tabularx}

\usetheme{Warsaw}
\mode<presentation>
\setbeamertemplate{navigation symbols}{}

\hypersetup{pdfpagemode=FullScreen, colorlinks=true,
  pdftitle={Présentation projet méthodes approchées.}, backref}

\title{Projet de méthodes approchées}

\institute{M1 informatique UM2}
\author{Dyce William, Loukil Amal, Ouazzani Sabrina}
\date{semestre 2 : 2011-2012}
\begin{document}

\begin{frame}
  \titlepage
\end{frame}

\section{Introduction}

\begin{frame}
  \frametitle{Problème}
  \begin{itemize}
  \item problèmes d'optimisation 
    \begin{itemize}
    \item[] $\nearrow$ problèmes faciles
    \item[] $\searrow$ \fbox{problèmes NP-difficiles}
    \end{itemize}
  \item  $\Rightarrow$ méthodes exactes 
    \begin{itemize}
    \item[] $\nearrow$ programmation dynamique
    \item[] $\searrow$ branch and bound
    \end{itemize}

    
  \item  $\Rightarrow$ méthodes approchées
    \begin{itemize}
    \item[] $\longrightarrow$ algorithmes d'approximation
    \end{itemize}
  \end{itemize}
\end{frame}

\begin{frame}
  \frametitle{Table of Contents}
  \begin{columns}
    \begin{column}[]{5cm}
      \tableofcontents
    \end{column}
    \begin{column}[]{5cm}
      \begin{center}
        \includegraphics[height=3.5cm]{commerce.png}
      \end{center}
    \end{column}
  \end{columns}
\end{frame}

\section{Programmation dynamique}

\begin{frame}
  \frametitle{Outils}
  \begin{columns}
    \begin{column}[]{5cm}
      \begin{center}
        \includegraphics[height=3.5cm]{tools.jpeg}
      \end{center}
    \end{column}
      \begin{column}[]{5cm}
         \begin{block}{Programmation}
        langage C
      \end{block}
      \begin{block}{Tests}
        \begin{itemize}
        \item tests unitaires
        \item tests de temps d'exécution
        \end{itemize}
      \end{block}
      \end{column}
    \end{columns}
  \end{frame}

\begin{frame}
  \frametitle{Partition}
  \begin{alertblock}{Formules}
    \begin{equation}
      \begin{cases}
        tab[0] = 1; \\
        si ( tab[j] == 1 ) {
          tab[j + poids[i]] = 1;
        } \\
      \end{cases}
    \end{equation}
  \end{alertblock}
\end{frame}

\begin{frame}
  \frametitle{Partition}
  \begin{columns}
    \begin{column}[]{5cm}
      \begin{center}
        \includegraphics[height=3.5cm]{unitest.jpg}
      \end{center}
    \end{column}
      \begin{column}[]{5cm}
        \begin{block}{Tests unitaires}
        \end{block}
      \end{column}
    \end{columns}
  \end{frame}
  
  \begin{frame}
    \frametitle{Partition}
    \begin{block}{Jeux d'essai}
    \end{block}
  \end{frame}
  
  \begin{frame}
    \frametitle{Partition : courbe}
  \end{frame}
  
  \begin{frame}
    \frametitle{Sac à dos}
    \begin{alertblock}{Formules}
      \begin{equation}
        \begin{cases}
          tab[0][w] = 0 \\
          tab[i][j] = max(tab[i-1] [j], (tab[i-1] [j-poids[i]] + utilite[i])); \\
        \end{cases}
      \end{equation}
    \end{alertblock}
  \end{frame}

\begin{frame}
  \frametitle{Sac à dos}
  \begin{columns}
    \begin{column}[]{5cm}
      \begin{center}
        \includegraphics[height=3cm]{Knapsack.png}
      \end{center}
    \end{column}
      \begin{column}[]{5cm}
        \begin{block}{Tests unitaires}
        \end{block}
      \end{column}
    \end{columns}
  \end{frame}


  \begin{frame}
    \frametitle{Sac à dos}
    \begin{block}{Jeux d'essai}

    \end{block}
  \end{frame}

  \begin{frame}
    \frametitle{Sac à dos : courbe}
  \end{frame}

  \begin{frame}
    \frametitle{Voyageur de commerce}
    \begin{alertblock}{Formules}
      \begin{equation}
        \begin{cases}
          C[S][i] = poids[0][i] \text{si S ne contient que $0$} \\
          C[S][i] = min_{k \in S - \{ 0 \}} \{ C[S- \{ k \}][k] + poids[k][i]  \}
        \end{cases}
      \end{equation}
    \end{alertblock}
  \end{frame}

\begin{frame}
  \frametitle{Voyageur de commerce}
  \begin{columns}
    \begin{column}[]{5cm}
      \begin{center}
        \includegraphics[height=4cm]{salesman2.jpg}
      \end{center}
    \end{column}
      \begin{column}[]{5cm}
        \begin{block}{Tests unitaires}
        \end{block}
      \end{column}
    \end{columns}
  \end{frame}

\begin{frame}
\frametitle{Voyageur de commerce}
\begin{block}{Jeux d'essai}
\end{block}
\end{frame}

\begin{frame}
\frametitle{Voyageur de commerce : courbe}
\end{frame}

\section{Branch and Bound TSP}

\begin{frame}
  \frametitle{Branch and Bound TSP}
  \begin{block}{Outils}
    \begin{itemize}
    \item GLPK
    \item langage C/C++
    \end{itemize}
  \end{block}
  \begin{block}{Structures de données}
  \end{block}
\end{frame}

\begin{frame}
  \frametitle{Solution initiale}
  \begin{itemize}
  \item chaîne de poids le plus faible~;
  \item voisinage 2--opt~;
  \item voisinage 3--opt.
  \end{itemize}
\end{frame}


\begin{frame}
  \frametitle{Comparaisons}
\end{frame}

\begin{frame}
  \frametitle{Complexité théorique}
\end{frame}

\begin{frame}
  \frametitle{Jeux d'essais}
\end{frame}

\begin{frame}
  \frametitle{Temps d'exécution}
\end{frame}

\section{Algorithme $\frac{3}{2}$ TSP}

\begin{frame}
\frametitle{$\frac{3}{2}$ approximation TSP}
\begin{block}{Outils}
  \begin{itemize}
  \item langage C
  \item bibliothèque Boost Graph
  \item gprof
  \end{itemize}
\end{block}
\begin{block}{Structures de données}
\end{block}
\end{frame}

\begin{frame}
  \frametitle{Complexité théorique}
\end{frame}

\begin{frame}
  \frametitle{Jeux d'essais}
\end{frame}

\begin{frame}
  \frametitle{Temps d'exécution}
\end{frame}

\section{Comparaisons}

\begin{frame}
  \frametitle{Comparaisons TSP}
  
\end{frame}

\section{Conclusion}
\begin{frame}

\end{frame}


\begin{frame}
  \begin{center}
    Merci pour votre attention.
  \end{center}
\end{frame}

\end{document}
