\documentclass[french]{beamer}
%\usepackage{listingsutf8}
%\usepackage[T1]{fontenc}
%\usepackage[utf8]{inputenc}
\usepackage{fontspec}
\usepackage[french]{babel}
\usepackage{amsmath, amssymb, mathrsfs}
\usepackage{graphicx}
\usepackage{listings}
\usepackage{tikz}
\usepackage{pgfplots}
\usepackage{epic, eepic}
\usepackage{array, multirow, tabularx}

\usetheme{Warsaw}
\mode<presentation>
\setbeamertemplate{navigation symbols}{}

\hypersetup{pdfpagemode=FullScreen, colorlinks=true,
  pdftitle={Présentation projet méthodes approchées.}, backref}

\title{Projet de méthodes approchées}

\institute{M1 informatique UM2}
\author{Dyce William, Loukil Amal, Ouazzani Sabrina}
\date{semestre 2 : 2011-2012}
\begin{document}

\begin{frame}
\titlepage
\end{frame}

\section{Introduction}

\begin{frame}
\frametitle{Problème}
\end{frame}

\begin{frame}
\frametitle{Table of Contents}
\begin{columns}
\begin{column}[]{7cm}
\tableofcontents
\end{column}
\begin{column}[]{2cm}
%\includegraphics[height=3cm]{cantor.jpg}
\end{column}
\end{columns}
\end{frame}

\section{Programmation dynamique}

\begin{frame}
\frametitle{Principe}
\begin{alertblock}{Formules}
\end{alertblock}

\begin{block}{Programmation}
langage C
\end{block}

\end{frame}


\begin{frame}
\frametitle{Partition}
\begin{block}{Jeux d'essai}

\end{block}
\end{frame}


\begin{frame}
\frametitle{Sac à dos}
\begin{block}{Jeux d'essai}

\end{block}
\end{frame}

\begin{frame}
\frametitle{Voyageur de commerce}
\begin{block}{Jeux d'essai}

\end{block}
\end{frame}

\section{Branch and Bound}

\begin{frame}
\frametitle{Outils}
\begin{block}{Langage}
\begin{itemize}
\item GLPK
\item langage C/C++
\item bibliothèque
\end{itemize}
\end{block}
\end{frame}

\begin{frame}
\frametitle{Solution ininitale}
\begin{itemize}
\item chaîne de poids le plus faible~;
\item voisinage 2--opt~;
\item voisinage 3--opt.
\end{itemize}
\end{frame}


\begin{frame}
\frametitle{Comparaisons}
\end{frame}

\begin{frame}
\frametitle{Algorithme $\frac{3}{2}$}
\end{frame}

\begin{frame}
\frametitle{Complexité théorique}
\end{frame}

\begin{frame}
\frametitle{Jeux d'essais}
\end{frame}

\begin{frame}
\frametitle{Temps d'exécution}
\end{frame}


\section{Comparaisons}


\begin{frame}
\frametitle{Jeux d'essais selon la solution initiale}

\end{frame}

\begin{frame}
\frametitle{Temps d'exécution selon la solution initiale}
\end{frame}

\section{Conclusion}
\begin{frame}

\end{frame}


\begin{frame}
\begin{center}
Merci pour votre attention.
\end{center}
\end{frame}

\end{document}
