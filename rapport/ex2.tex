\paragraph{Question 1}

Nous proposons la modélisation suivante en programmation linéaire en
nombres entiers pour le problème de la couverture d'ensembles~:
\begin{equation}
\begin{cases}
min \sum_{j=1}^m w_j s_j\\
\sum_{j=1}^{m} s_{j} \geq 1 \\
s_i \in \{0,1\} \\
\end{cases}
\end{equation}
avec $s_j = 1$ si l'ensemble $S_j$ est choisi, 0 sinon.

Proposition sous forme  d'écriture matricielle, avec M matrice
d'appartenance d'un élément $e_i$ à un ensemble $S_j$~:

\begin{equation}
\begin{cases}
min \sum_{j=1}^m w_j x_j\\
\forall i \in \{1, \dots n \} \sum_{j=1}^{m} M_{ij}x_{j} \geq 1 \\
\forall i \in \{1, \dots n\} x_i \in \{0,1\} \\
\end{cases}
\end{equation}
avec $x_j = 1$ si l'ensemble $S_j$ est choisi, 0 sinon.

\paragraph{Question 2}
 Dans un programme en nombre réels, on a plutôt$ s_j \in [0,1]$. Nous
 proposons donc la procédure d'arrondis suivante sur les $s_j$~:
\begin{itemize}
\item Si $s_j \geq \frac{1}{f}$ alors $s_j = 1$,
\item si $s_j < \frac{1}{f}$ alors $s_j = 0$.
\end{itemize}

\paragraph{Question 3}

La procédure d'arrondie précédente garantie une solution réalisable
car elle respecte toutes les contraintes de la modélisation énoncée à
la question 1.

\begin{proof}
TODO : william
\end{proof}

\paragraph{Question 4}

Montrons que le problème de la couverture d'ensemble possède un
algorithme $f$-approché.

\begin{proof}
Soit $C^*$ une solution optimale pour le problème de la couverture
d'ensemble, soit C la solution renvoyée par un algorithme
$f$-approché.

$C^* \geq  \sum_{i=1}^n w_{e_i}$ car deux éléments peuvent appartenir à un même sous-ensemble.
$C \leq f * \sum_{i=1}^n w_{e_i}$ car un élément appartient a au plus $f$ sous-ensembles \\

On a donc $\frac{C}{C^*} \leq f$.
\end{proof}

\paragraph{Question 5}

Quand $f_i = 2$, $e_i$ appartient à deux sous-ensembles. On retrouve
donc le problème de la couverture d'arêtes (les éléments) par des
sommets (les sous-ensembles) (vertex cover).
