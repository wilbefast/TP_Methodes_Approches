\paragraph{Question 1}

Les deux problèmes considérés sont NP-complets. Plus exactement, la
coloration de sommet est un problème non-APX tandis que la coloration
d'arêtes est un problème APX. 

\paragraph{Question 2~:}
\begin{itemize}
\item Si $OPT(I) \leq 3$ alors on a \\
$\frac{A(I)}{OPT(I)} < \frac{4}{3}$ \\
d'où $A(I) < 4$.
\item Si $OPT(I) \geq 4$ alors on a \\
$\frac{A(I)}{OPT(I)} < \frac{4}{3}$ \\
d'où $A(I) \geq 3$.
\end{itemize}

\paragraph{Question 3~:}

\begin{proof}
Le problème de 3--coloration est NP-complet. Or, à partir de la
question 2, on peut décider si 3 couleur suffisent pour colorier les
sommets ou les arêtes de notre graphe. Donc il est absurde de supposer
qu'il existe un algorithme avec une performance relative strictement
inférieure à $\frac{4}{3}$.
\end{proof}

\subsection{Comparaison de méthodes}


